\documentclass[11pt]{amsart}
\usepackage{geometry}                % See geometry.pdf to learn the layout options. There are lots.
\geometry{letterpaper}                   % ... or a4paper or a5paper or ... 
%\geometry{landscape}                % Activate for for rotated page geometry
%\usepackage[parfill]{parskip}    % Activate to begin paragraphs with an empty line rather than an indent
\usepackage{graphicx}
\usepackage{multicol, hanging}
\usepackage{amssymb}
\usepackage{epstopdf}
\usepackage{supertabular}
\usepackage{hanging}
\DeclareGraphicsRule{.tif}{png}{.png}{`convert #1 `dirname #1`/`basename #1 .tif`.png}

\title{Simulating Language Lecture II}
\author{Simon Kirby \& Andrew Smith}
\date{January 14$^{th}$, 200}                                           % Activate to display a given date or no date

\begin{document}
\maketitle
\section{Lecture II: Modelling innate signalling systems}
\begin{itemize}
\item Starting with communication, not {it homo sapiens} language. 
\item Bacteria: quorum sensing - chemical swarm behavioural change. Communication. 
\item Interspecial communication: innate signalling.
\item Simple model: mapping between {\it meaning} and {\it signals}
\item \underline{Agent}: the computational equivalent of an individual being studied. 
\item Mapping of signals: how does it evolve, what happens when two agents have similar maps
\item \underline{Function}: x $\rightarrow$ y
\item Another way of storing maps is to store the communication system within a matrix
\begin{center}
\begin{supertabular}{|c|c|c|c|} 
\hline
\multicolumn{4}{|c|}{Vervet Matrix} \\ \hline
 	& s1 & s2 & s3 \\ \hline
m1 & 1 & 0 & 0 \\
m2 & 0 & 1 & 0 \\
m3 & 0 & 0 & 1 \\ 
\hline 
\end{supertabular} \end{center}
\item could also have larger numbers, could be done differently. for instance, all 1 in the s1 column.
\item either always pick the highest, or pick proportionately. designer needs to choose.
\item biggest number always: winner-take-all scenario.
\begin{center}
\begin{supertabular}{|c|c|c|c|} 
\hline
\multicolumn{4}{|c|}{Vervet Matrix} \\ \hline
 	& s1 & s2 & s3 \\ \hline
m1 & 1 & 2 & 0 \\
m2 & 0 & 1 & 1 \\
m3 & 0 & 3 & 4 \\ 
\hline 
\end{supertabular} \end{center}
\item The above has a communicative accuracy level of .33 (run 10k times) (two agents, same matrix)
\item in meaning one, always choose s2. for meaning 2, choose s2 or s3...and so on. 
\item \underline{production}: look along rows and pick highest
\item \underline{reception}: look along rows and pick highest
\item mx $\rightarrow$ Sy $\rightarrow$ Mz : if this happens, it is a good system
\item To find out, get two agents, same system, and run proportional stats on the amount of times meanings are the same. That's how you rate a system.
\item \underline{Monte Carlo approach}: doing things repeatedly to find proportional rate of good. (not proper syntax). What you're looking for is measure of \underline{communicative accuracy}.


%\begin{multicols}{2}{\begin{hangparas}{.5cm}{1}\noindent
%\end{hangparas} }\end{multicols}

\end{itemize}

\end{document}  