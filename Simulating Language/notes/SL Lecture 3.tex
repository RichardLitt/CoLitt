\documentclass[11pt]{amsart}
\usepackage{geometry}                % See geometry.pdf to learn the layout options. There are lots.
\geometry{letterpaper}                   % ... or a4paper or a5paper or ... 
%\geometry{landscape}                % Activate for for rotated page geometry
%\usepackage[parfill]{parskip}    % Activate to begin paragraphs with an empty line rather than an indent
\usepackage{graphicx}
\usepackage{multicol, hanging}
\usepackage{amssymb}
\usepackage{epstopdf}
\usepackage{supertabular}
\usepackage{hanging}
\DeclareGraphicsRule{.tif}{png}{.png}{`convert #1 `dirname #1`/`basename #1 .tif`.png}

\title{Simulating Language Lecture 5}
\author{Simon Kirby \& Andrew Smith}
\date{January 20$^{th}$, 2000}               

\begin{document}
\maketitle
\section{Lecture III: Python}
\begin{itemize}
\item Computational models allow us to bridge between theory and prediction for understanding complex dynamic systems with many interacting components
\item first example: communication in animals wit innate signalling systems
\item tread signalling system as a mapping between a fixed set of meanings and a fixed set of signals %controversial
\item modelled as an innately- determined matrix of weighted associations
\item different matricies give different production and reception behaviours. 
\item communicative accuracy for a pair of speaker and hearer can be defined as the proportion of utterances where hearer converges on same meaning as speaker. %succesful means the at the same meaning is arrived at. 
%On monday we looked at populations of agents, and how good they are. Tslolam. 
\end{itemize}
Diggression to Worksheet 1 and worksheet solutions
Where do these signalling matrices come from?
\begin{itemize}
\item Innately specified for vervets, therefore result of genes
\item How would an organism end up with a set of genes that gives them a good communicative accuracy score?\\
	- must have been natural selection and all that
\item {\bf Theory:} natural selection will give us organisms with genes that specify signalling systems which have high communicative accuracy
\item How can we sure that this is right? Modelling. 
\end{itemize}
Modelling Evolution
\begin{itemize}
\item Many ways of doing this in computational simulation. One approach: {\it genetic algorithms} (see reading for this week - Mitchell, 1998)
\item Key ingredients: genotype (genes) and phenotype (organism)
\item Also need these key ingredients:
\begin{itemize}
\item a population of organisms
\item a task they are trying to succeed at
\item  a measure of how fir theey are it this task
\item a way of selecting the fittest
\item  a way of allowing the genes to survive
\item and something else
\end{itemize}
\end{itemize}
\newpage
Our Model:
\begin{itemize}
\item simplify things a bit: trat genes and phenotype as equivalent and get rid of dimorphism %would be interesting to track what would happen if you went back in
\item The simulation:
\begin{itemize}
\item create a population of random signal matrices [what makes a random signal matrix?]
\item assess each member of population for fitness [how do we measure fitness?]
\item pick a parent based on fitness [How do we pick a parent?]
\item copy parent, with a chance of mutation, to create new offspring [how does mutation work?]
\item do 3 \& 4 enough times to come up with a new population that's the same size as the old one
\item replace old pop. with a new one [How is this done? 
\item repeat steps 2 to 6 many times
\end{itemize}
\end{itemize}
Main Research Question: \begin{itemize}
\item Under what conditions will we see the emergence of "optimal" communication systems? (ie. when will we see a stable population of agents in which any pair of agents would have a communicative accuracy of 1.0)
\item Main parameter: how do we assess fitness?
\item {\bf What is the {\it fitness function?}}
\item Key considertions: 
\begin{itemize} \item How do we pick communicative partners
\item who gets rewarded for successful communication?
\end{itemize}
\end{itemize}

Readings:
\begin{itemize}
\item Oliphant, M. (1996) THe dilemma of Saussurean Communication. {\it Biosystems}, 37:31-38.
\item Mitchell, M. (1998) An introduction to genetic algorithms. pp. 1-16
\end{itemize}

















\end{document}  