%\requirepackage{lineno}
\documentclass[10pt]{article}
\usepackage[left=4cm,top=1cm,right=3cm,left=3cm,nofoot]{geometry}  \geometry{a4paper}                   
\usepackage{tipa, apalike, graphicx, amssymb, delarray, epstopdf, amsmath, amsthm, setspace, supertabular, qtree, hyperref, footnote, palatino, url, multicol, hanging, fullpage, supertabular, ulem} 
%, lineno, gb4e} % gb4e messes up math mode. 
\clubpenalty=300 \widowpenalty=300
\DeclareGraphicsRule{.tif}{png}{.png}{`convert #1 `dirname #1`/`basename #1 .tif`.png}
%\usepackage[parfill]{parskip} 			%Activate for line, not indent, paragraphs.
\newenvironment{itemise}{
\begin{itemize}
  \setlength{\itemsep}{1pt}
  \setlength{\parskip}{0pt}
  \setlength{\parsep}{0pt}
}{\end{itemize}}

\title{Homework 2}
\author{Richard Littauer, Grady Payson}
\date{\today}                          		% Activate to display a given date or no date
\singlespace
\begin{document}
\maketitle

\section*{Research Question} 
Does Red Bull (a caffeinated, high-energy drink) consumption cause a higher spoken word per minute count in casual conversation?
\section*{Experimental Strategy}
For this experiment, we gathered data from 20 participants. These participants were paired together, and instructed to talk casually about interesting topics (with occasional leads) for twenty minutes. After this, they had a two minute break, were given a shot of 100ml of Red Bull, and then were instructed to repeat talking for ten minutes. This was repeated until the participants had drunk nineteen shots of 100ml of Red Bull. The words per minute average was then counted up for each of the ten minute segments, and the differences for this average were used to show causation. 

All tests were started at 10:00am, and the participants were instructed to get a good night's sleep, and have no stimulants before hand. All speakers were male students in their twenties. All were Native Wompanoag speakers, and the conversations were all done in Wompanoag.
\\ \ \\
The entire experiment was completely hypothetical, and all data has been massaged and tampered with.

\section*{Homework Questions}
\begin{enumerate}
\item {\bf Generate data:} Done, attached.
\item {\bf Organise the data:} Done. 
\item {\bf Dependent variables:} The words per minute. This was measured per participant per the amount of Red Bull drunk. {\bf Scale of measurement:} Ratio scale - 0 being complete and awkward silence, and 637 being the world fast-talking record. 
\item {\bf Independent variables:} The amount of Red Bull being drunk. This is the only variable we change throughout the experiment. There are twenty possible values; the amount of drinks drunk up to that point. 
\item {\bf Null hypothesis:} The Null Hypothesis, given the above wording, would be that no, there is no change to the rate of speech when drinking a caffeinated beverage Red Bull, or that they speak slower. {\bf Alternative hypothesis:} The Alternative hypothesis is that there was a change of speech rate, and that it was faster than the base word per minute rate for each individual. 
\item {\bf Type I error:} This error would happen if the speech rate was not significantly faster, but the null hypothesis was rejected anyway. {\bf Type II error:} This would happen if the speech rate was in fact significantly faster, but the conclusion drawn was that there was no change. 
\end{enumerate}





















%\bibliographystyle{apasoft}
%\bibliography{/Users/richardlittauer/Desktop..}
\end{document}