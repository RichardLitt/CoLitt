\documentclass[11pt]{article}
\usepackage{eacl2012}
\usepackage{times}
\usepackage{latexsym}
\usepackage{amsmath}
\usepackage{multirow}
\usepackage{url}
\DeclareMathOperator*{\argmax}{arg\,max}
\setlength\titlebox{6.5cm}    % Expanding the titlebox

\title{Visualising Typological Relations: Using Heat Maps with WALS}
%Title should perhaps be changed. 

\author{Richard Littauer \\
University of Saarland\\
Computational Linguistics Department\\
66041 Saarbr\"ucken, Germany\\
  {\tt richard.littauer@gmail.com} \\\And
Alexis Palmer\\
University of Saarland\\
Computational Linguistics Department\\
66041 Saarbr\"ucken, Germany\\
  {\tt apalmer@coli.uni-sb.de} \\\AND
Rory Turnbull \\
Ohio State University\\
Department of Linguistics\\
Columbus, Ohio\\
  {\tt turnbull@ling.osu.edu} \\}

\date{}

\begin{document}
\maketitle
\begin{abstract}
This paper presents a novel way of visualising linguistic typological data. Computational methods have only recently been applied in the formation and use of large typological databases. Many studies have since focused on discovering relations between languages using typology, often using sophisticated statistical techniques. However, few papers have provided new ways of visually presenting the resulting data. Here, we show that one can use the data from the World Atlas of Language Structures\cite{wals-2011} to develop both heat maps and admixture graphs that visually show the interconnected relationships between languages and language families. We hope that the images will bring a new perspective to the data, resulting in interesting findings and illuminating areas of research.
\end{abstract}

%% This outline is just a suggestion. Feel free to mix it up as much as you want if you feel it would or should be better another way. 

\section{Introduction}
\subsection{Heat maps}
\begin{itemize}
\item brief history of typology
\item WALS
\item visualisation
\item what visualisation can do for us
\item Heat Maps and Admixture-  where they come from, what they are doing here
\item plan for the paper
\end{itemize}


%
%
%
%Relations:
%    Languages are related either phylogenetically, or geographically. There is
%    also a possible Contact option, but we do not have the proper typological
%    details for this, nor historic details.
%
%    Relations are defined by features. A possibility is combining features, and
%    creating a feature similarity metric - for instance, aligning various types
%    of syntactic universals together.
%
%    Here, it is possible to sort phylogenetically by WALS hierarchies, or by
%    ethnologue hierarchies. There is no real difference between them. Another
%    option would be to use MultiTree for identified trees. WALS data can be
%    sorted by Family or Genus, not yet by subfamily. Ethnologue data can
%    currently only be calibrated by the overarching parents, and not by
%    sub-roots, although this would be useful.
%
%


\section{Material and Methods}
\subsection{Linguistic material}
\begin{itemize}
\item Information about WALS data
%Cleaning:
%    The WALS data can be cleaned with the data_clean() function. It should be
%    noted that WALS is sparse, with only 76492 datapoints for 2678 languages,
%    which, at 144 features, is only 19% full. This may not even be the right
%    way of seeing it, however. 
\begin{itemize} \item Description of typological data
\item Longitude and latitude, and how it is measured
%Distance:
%    For each language, the distance has been calculated. Currently we have a
%    2678x2678 matrix, which is a little large.
%
%    Geographical distances can be calculated by radius distance from the
%    centroid, or the amount of languages specified that pass the cleaning test
%    and are closes to the centroid. 
\item language families
\item WALS-language Sparseness \end{itemize}
\item Information about Multitree\cite{multitree}
%Multitree - for relations, can code out the .xml and use those. http://multitree.linguistlist.org/about
\item Information about Ethnologue\cite{ethnologue} scrape %provided by steven moran -need to get this
\end{itemize}

\subsection{Methods}
\begin{itemize}
\item Cleaning WALS data
\item Collapsing WALS data
\item Scraping Ethnologue, Multitree, formati
\item Measuring Phylogenetic distance
\item Measuring Geographical distance
%Haversine Formula - distance on a sphere (earth) http://en.wikipedia.org/wiki/Haversine_formula
% (Vincenty formula would be more accurate (as earth is not a 
% perfect sphere), but much more computationally intense for
% 2678*2678 comparisons.
%   http://www.codecodex.com/wiki/Calculate_Distance_Between_Two_Points_on_a_Globe#Python
\item Combined map

%Ordering:
%    For vector relations, we have an issue where to center the matrix.
%    If centred on a point, both ends may be close or far apart - one
%    can only stare at the center of the graph for information.
%
%    We could sort them lattitudinally or longitudinally - consider the case of
%    a Dravidian language at the south of India. This would work well, here. Or
%    Patagonian. Japanese would work - but we don't have the dialect scales, so
%    this may not show us anything, especially not with WALS. This helps us in
%    forcing our hand to make smaller maps - say, 30 languages - instead of
%    massive ones. 

%Should be possible - give distance equal weightings, find the aggregate. 
\item Compiling python scripts, converting into R
\item Sorting R output
\end{itemize}

\section{Results} %Or visualisations?
\subsection{Heat maps}
\begin{itemize}
\item Phylogenetic distance heatmaps
%only relevant ones, or ones that seem particularly useful. We should be making a few - world, region, etc. Going to need to make a supplement, perhaps. 
\item Geographical distance heatmaps
\item Combined maps
\end{itemize}
\subsection{Heat maps}
\begin{itemize}
\item Admixture mappings
%families
%regions
%combined
\end{itemize}

\section{Discussion}
\begin{itemize}
\item What these tell us (map by map)
\item What these tell us, overall - implications
\item Warnings: sparse data, data not there, etc. 
\item Future work 
%    We could also plot on a world map, but that would not be a heat map. An
%    option for this would be to color gradiantly based on certain featurs - for
%    instance, color could indicate phoneme size. 
%
%    Another option would be a matrix map of some sort - Again, not a heat map,
%    and beyond the scale of this current study.
\end{itemize}

\bibliographystyle{acl}
\bibliography{lingviz}
%WALS
%Multitree
%Ethnologue

\end{document}
