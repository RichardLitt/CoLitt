\documentclass[11pt]{article}
\usepackage{eacl2012}
\usepackage{times}
\usepackage{latexsym}
\usepackage{amsmath}
\usepackage{multirow}
\usepackage{url}
\DeclareMathOperator*{\argmax}{arg\,max}
\setlength\titlebox{6.5cm}    % Expanding the titlebox

\title{Visualising Typological Relations: Using Heat Maps with WALS}
%Title should perhaps be changed. 

\author{Richard Littauer \\
University of Saarland\\
Computational Linguistics Department\\
Saarbr\"ucken, Germany\\
  {\tt richard.littauer@gmail.com} \\\And
Rory Turnbull \\
Ohio State University\\
Department of Linguistics\\
Columbus, Ohio\\
  {\tt turnbull@ling.osu.edu} \\\AND
Alexis Palmer\\
University of Saarland\\
Computational Linguistics Department\\
Saarbr\"ucken, Germany\\
  {\tt apalmer@coli.uni-sb.de}\\} 

\date{}

\begin{document}
\maketitle
\begin{abstract}
This paper presents a novel way of visualising linguistic typological data. Computational methods have only recently been applied in the formation and use of large typological databases. Many studies have since focused on discovering relations between languages using typology, often using sophisticated statistical techniques. However, few papers have provided new or newly applied ways of visually presenting the resulting data. Here, we show that one can use the data from the World Atlas of Language Structures\cite{wals-2011} to develop heat maps that can visually show the interconnected relationships between languages and language families. We hope that the images will bring a new perspective to the data, resulting in interesting findings and illuminating areas of research.
\end{abstract}

%% This outline is just a suggestion. Feel free to mix it up as much as you want if you feel it would or should be better another way. 

\section{Introduction}
.[Filler][Filler]

%\item brief history of typology
Typology has been used to derive implications about possible languages, and about the ordering of the human mind. Greenberg (1963) \nocite{greenberg} is famous for finding many cross-linguistic typological implications, such as %example,
, and claiming that they represented universals of grammar, and that they would hold true for all languages. In a similar vein, Chomsky (2000) \nocite{chomsky}%Check citation, make sure it is a good one
has argued as a generativist that typological universals show that language is clearly an artefact of constraints involved with the learning process, and that some typological features are impossible due to this. Dunn {\it et al} (2011) %Find citation
argued that a languages typology relies upon the previous generations' language more than biological, enironmental or cognitive constraints, and that there are pathways which are generally followed in language change based on the previous parent language. 

Outside of the scope of theory, language typology is useful for other reasons. %Cite Xia and Lewis, MT, NLP, etc....

%\item WALS
Towards this end, the World Atlas of Language Structures (WALS) was formed. WALS is "a large database of structural (phonological, grammatical, lexical) properties of languages gathered from descriptive materials (such as reference grammars) by a team of 55 authors (many of them the leading authorities on the subject)."(\url{http://www.wals.info}) At the time of submission, there were 81828 datapoints for 2678 languages (an average of 28 per language). At least one feature, the most populated, had data for 1519 languages. There were 144 different chapters, each containing values for different, related features. It can be seen easily from these numbers that the data on WALS is {\it sparse}. Ignoring the fact that a language having certain features will cancel out the possibility or probability of others, that means that the WALS possible data is only 15.8\% represented in the database.

Dealing with sparse data is a computational problem, as any statistical information drawn from the database will, to a large extent, be an artefact of the database. For instance, if half of the languages were marked as having uvular stops (unlikely), and then in reality if all other languages not in the database had uvular stops, then the knowledge we would glean from the database would be significantly false and misleading. Given that there are around 6,000 languages in the world, the amount of languages on WALS means that this is a serious concern. Many researchers in recent years have been developing work-arounds for sparse databases; often because languages with low resources have a similar problem. %Cite cite cite

A solution to dealing with this issue is to visualise the data in WALS, instead of relying on statistics.

%Alexis [filler][filler]

%\item visualisation
%\item what visualisation can do for us
%\item Heat Maps  where they come from, what they are doing here

%In this paper:
In this paper, we will first discuss more fully our starting data, before going on to discuss the problems with analysing this data and how our methodology dealt with them. We will then present several graphs that highlight the possibilities of graphing WALS data. 






\section{Material and Methods}
\subsection{Linguistic material}
\begin{itemize}
\item Information about WALS data

\begin{itemize} \item Description of typological data

%
%
%
%Relations:
%    Languages are related either phylogenetically, or geographically. There is
%    also a possible Contact option, but we do not have the proper typological
%    details for this, nor historic details.
%
%    Relations are defined by features. A possibility is combining features, and
%    creating a feature similarity metric - for instance, aligning various types
%    of syntactic universals together.
%
%    Here, it is possible to sort phylogenetically by WALS hierarchies, or by
%    ethnologue hierarchies. There is no real difference between them. Another
%    option would be to use MultiTree for identified trees. WALS data can be
%    sorted by Family or Genus, not yet by subfamily. Ethnologue data can
%    currently only be calibrated by the overarching parents, and not by
%    sub-roots, although this would be useful.
%
%

\item Longitude and latitude, and how it is measured
%Distance:
%    For each language, the distance has been calculated. Currently we have a
%    2678x2678 matrix, which is a little large.
%
%    Geographical distances can be calculated by radius distance from the
%    centroid, or the amount of languages specified that pass the cleaning test
%    and are closes to the centroid. 
\item language families
\item WALS-language Sparseness \end{itemize}
\item Information about Multitree\cite{multitree}
%Multitree - for relations, can code out the .xml and use those. http://multitree.linguistlist.org/about
\item Information about Ethnologue\cite{ethnologue} scrape %provided by steven moran -need to get this
\end{itemize}

\subsection{Methods}
\begin{itemize}
\item Cleaning WALS data
\item Collapsing WALS data
\item Scraping Ethnologue, Multitree, formati
\item Measuring Phylogenetic distance
\item Measuring Geographical distance
%Haversine Formula - distance on a sphere (earth) http://en.wikipedia.org/wiki/Haversine_formula
% (Vincenty formula would be more accurate (as earth is not a 
% perfect sphere), but much more computationally intense for
% 2678*2678 comparisons.
%   http://www.codecodex.com/wiki/Calculate_Distance_Between_Two_Points_on_a_Globe#Python
Each final list was then resorted, so that the source language was centered in the map. This was due to one of the primary issues with using distance on a two dimensional graph. On the one hand, we would want close languages to be close together on the heat map. However, given the source language Egyptian Arabic, this would mean that Zulu, Saami, Mohawk, and Japanese might all roughly be placed next to each other on the map. This also means that Japanese might be placed next to Mohawk, and Mohawk again next to Korean. This is not ideal, and was the main justification for limiting the sphere of possible geographical languages to a reasonable distance, given the data. 
\item Combined map

%Ordering:
%    For vector relations, we have an issue where to center the matrix.
%    If centred on a point, both ends may be close or far apart - one
%    can only stare at the center of the graph for information.
%
%    We could sort them lattitudinally or longitudinally - consider the case of
%    a Dravidian language at the south of India. This would work well, here. Or
%    Patagonian. Japanese would work - but we don't have the dialect scales, so
%    this may not show us anything, especially not with WALS. This helps us in
%    forcing our hand to make smaller maps - say, 30 languages - instead of
%    massive ones. 

%Should be possible - give distance equal weightings, find the aggregate. 
\item Compiling python scripts, converting into R
\item Sorting R output
\end{itemize}

\section{Results} %Or visualisations?
\subsection{Heat maps}
\begin{itemize}
\item Phylogenetic distance heatmaps
%only relevant ones, or ones that seem particularly useful. We should be making a few - world, region, etc. Going to need to make a supplement, perhaps. 
\item Geographical distance heatmaps
\item Combined maps
\end{itemize}


\section{Discussion}
\begin{itemize}
\item What these tell us (map by map)
\item What these tell us, overall - implications
\item Warnings: sparse data, data not there, etc. 
\item Future work 
%    We could also plot on a world map, but that would not be a heat map. An
%    option for this would be to color gradiantly based on certain featurs - for
%    instance, color could indicate phoneme size. 
%
%    Another option would be a matrix map of some sort - Again, not a heat map,
%    and beyond the scale of this current study.
\end{itemize}

\bibliographystyle{acl}
\bibliography{lingviz}
%WALS
%Multitree
%Ethnologue

\end{document}
