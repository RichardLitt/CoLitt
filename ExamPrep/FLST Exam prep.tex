%\requirepackage{lineno}
\documentclass[11pt]{article}
\usepackage[left=4cm,top=3cm,right=3cm,left=3cm,nofoot]{geometry}  \geometry{a4paper}                   
\usepackage{tipa, apalike, graphicx, amssymb, delarray, epstopdf, amsmath, amsthm, setspace, supertabular, qtree, hyperref, footnote, palatino, url, multicol, hanging, fullpage, supertabular, ulem} %, lineno, gb4e} 
%gb4e messes up math mode. 
\clubpenalty=300 \widowpenalty=300
\DeclareGraphicsRule{.tif}{png}{.png}{`convert #1 `dirname #1`/`basename #1 .tif`.png}
%\usepackage[parfill]{parskip} 			%Activate for line, not indent, paragraphs.

\newenvironment{pit}{
\begin{itemize}
  \setlength{\itemsep}{1pt}
  \setlength{\parskip}{0pt}
  \setlength{\parsep}{0pt}
}{\end{itemize}}


\title{FSLT Examination Preparation}
\author{Richard Littauer}		% Add your name here
\date{\today}                          		% Activate to display a given date or no date
\singlespace
\begin{document}
\maketitle

%The point is to have a latex analysis of the slides or relevant facts to know for each section, and subsequently, each slide. We can work on this collaboratively - remember to push and pull the sections you're working on, and try to make sure that you aren't clashing with other's doing the same work. 

\section{Map of the Field (Uzkoreit) 24.10-26.10}
% Richard is currently working on this section. 

\section{Linguistics Foundation (Delogu) 28.10-09.11}
%lamar is working on this section
\begin{itemize}
 \item Linguistic competence -The implicit knowledge a language user has of his language, which enables him to produce and understand any possible sentence of that language
 \item Linguistic performance - The actual use of language in real situations, which is conditioned by physiological and psychological constraints (memory limitations, shifts of attention, etc.)
\end{itemize}

\begin{enumerate}
 \item
\end{enumerate}
\section{Cognitive Foundations (Crocker) 14.11-18.11}

\section{Technological Foundations (Busemann) 21.11-25.11}

\section{Finite State Methods for Lexicon \& Morphology (Kiefer) 28.11-02.12}

\section{Parsing (Zhang) 05.12-09.12}

\section{Statistical NLP (Language Models) (Klakow, Wiegand) 12.12-23.12}

\section{Prosodic Models for Speech Technology (Moebius) 09.01}

\section{Speech Synthesis (Moebius, Lasarcyk) 11.01-16.01}

\section{Automatic Speech Recognition (Moebius) 18.01}

\section{Corpora for speech technology (Lasarcyk) 20.01}

\section{Semantics (Pinkal) 23.01-3.02}

\section{Discourse \& Dialogue (Sporleder) 06.02-08.02}


\end{document}