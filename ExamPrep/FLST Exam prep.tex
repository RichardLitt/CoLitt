%\requirepackage{lineno}
\documentclass[11pt]{article}
\usepackage[left=4cm,top=3cm,right=3cm,left=3cm,nofoot]{geometry}  \geometry{a4paper}                   
\usepackage{tipa, apalike, graphicx, amssymb, delarray, epstopdf, amsmath, amsthm, setspace, supertabular, qtree, hyperref, footnote, palatino, url, multicol, hanging, fullpage, supertabular, ulem} %, lineno, gb4e} 
%gb4e messes up math mode. 
\clubpenalty=300 \widowpenalty=300
\DeclareGraphicsRule{.tif}{png}{.png}{`convert #1 `dirname #1`/`basename #1 .tif`.png}
%\usepackage[parfill]{parskip} 			%Activate for line, not indent, paragraphs.

\newenvironment{itemise}{
\begin{itemize}
  \setlength{\itemsep}{1pt}
  \setlength{\parskip}{0pt}
  \setlength{\parsep}{0pt}
}{\end{itemize}}


\title{FSLT Examination Preparation}
\author{Richard Littauer}		% Add your name here
\date{\today}                          		% Activate to display a given date or no date
\singlespace
\begin{document}
\maketitle
\tableofcontents

\section*{Overview}
%The point is to have a latex analysis of the slides or relevant facts to know for each section, and subsequently, each slide. We can work on this collaboratively - remember to push and pull the sections you're working on, and try to make sure that you aren't clashing with other's doing the same work. 
Format of the exam:
\begin{itemise}
\item 9 short questions for each section. These are obligatory. 
\item 5 long questions from 6 sections. You can pick up 3 from them. 
\begin{itemise}
\item linguistic foundations
\item finite-state automata
\item parsing
\item seminar
\item statistical NLP
\item speech
\end{itemise}
\end{itemise}
\newpage
\section{Map of the Field (Uzkoreit) 24.10-26.10}

Was anything actually covered in these two sessions? Does anyone remember? 

\section{Linguistics Foundation (Delogu) 28.10-09.11}
%lamar is working on this section

\subsection{Linguistic Foundations I}
- The difference between linguistic competence and linguistic performance
\begin{itemize}
 \item \textbf{Linguistic competence} -The implicit knowledge a language user has of his language, which enables him to produce and understand any possible sentence of that language
 \item \textbf{Linguistic performance} - The actual use of language in real situations, which is conditioned by physiological and psychological constraints (memory limitations, shifts of attention, etc.)
\end{itemize}
- The difference between sentence gramaticality and acceptability
\begin{itemize}
 \item \textbf{Gramaticality} - A sentence is gramatical if it is formed according to the gramar of the language
\item \textbf{Acceptability} - A sentence is acceptable if it 'sounds good' to a native speaker
\end{itemize}
note: a sentence can be gramatical but unnacceptale (e.g. hard to process)
\\\\
- Four definititions of \textbf{grammar}
\begin{itemize}
 \item The linguistic rules of a language that every native speaker intuitevly knows - Linguistic competence
\item A model of the linguistic rules that every speaker of a language intuitively knows - A theory of linguistic competence 
\item The rules and principles that describe the linguistic behavior of native speakers - Descriptive grammar
\item - The rules and principles that prescribe the linguistic behavior of native speakers, according to some authority - Prescreptive grammar
\end{itemize}
- Linguistic structuralism and the \textbf{inductive method} 
\begin{itemize}
 \item Structural linguistics thus involves collecting a corpus of utterances and then attempting to classify all of the elements of the corpus at their different linguistic levels
\item linguistic constituents were identified by the set of all contexts in which they can occur
\end{itemize}
- Chomsky and the \textbf{Generative Program}
\begin{itemize}
\item A rejection of structuralism and a redefinition of the object of investigation
\item Developed from two key observations:
  \begin{enumerate}
  \item Linguistic competence: people are able to understand and produce an infinite number of grammatical sentences
    \begin{itemize}
    \item The inductive method used by the Structuralists was inadequate (and unable to account for key linguistic phenomena like structural ambiguity)
    \end{itemize}
  \item Language acquisition: children are able to ‘learn’ their language perfectly, even though they are exposed to defective inputs
    \begin{itemize}
    \item The idea that language is learnt through stimulus-response processes, as argued by the Behaviorists (e.g., Skinner 1957) was no longer tenable
    \end{itemize}
  \end{enumerate}
\item The task of a linguist is comparable to a child’s acquisition of linguistic competence.
  \begin{itemize}
  \item Since children do not acquire language inductively, the new methodology must be deductive
  \end{itemize}
\item The object of investigation is no longer the product language as represented by corpora but the linguistic competence of native speakers
\item Linguistic competence is an internalized (mentally- represented) grammar capable of generating an infinite number of grammatical sentences and none of the ungrammatical ones
\item The goal of linguistics is to provide an adequate grammar as a model of linguistic competence
\end{itemize}

\subsection{Linguistic Foundations II}
\begin{itemize}
 \item \textbf{Morphology} - The study of word structure and word formation
\end{itemize}

\begin{enumerate}
 \item FOO
\end{enumerate}

\section{ Cognitive Foundations (Crocker) 14.11-18.11}

\begin{enumerate}
 \item
\end{enumerate}


%\section{Technological Foundations (Busemann) 21.11-25.11}
%
%\section{Finite State Methods for Lexicon \& Morphology (Kiefer) 28.11-02.12}
%
%\section{Parsing (Zhang) 05.12-09.12}
%
%\section{Statistical NLP (Language Models) (Klakow, Wiegand) 12.12-23.12}
%
%\section{Prosodic Models for Speech Technology (Moebius) 09.01}
%
%\section{Speech Synthesis (Moebius, Lasarcyk) 11.01-16.01}
%
%\section{Automatic Speech Recognition (Moebius) 18.01}
%
%\section{Corpora for speech technology (Lasarcyk) 20.01}
%
%\section{Semantics (Pinkal) 23.01-3.02}
%
%\section{Discourse \& Dialogue (Sporleder) 06.02-08.02}


\end{document}