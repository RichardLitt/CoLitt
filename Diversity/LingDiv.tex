Notes for the Linguistic Diversity Paper. 

Biodiversity hotspots:
http://en.wikipedia.org/wiki/Biodiversity_hotspot
http://www.biodiversityhotspots.org/

Different levels:

internal vs external data. 

Semantic bit-sharing - semantic field sharing - semantic overlap/identical fields
Lingueme - phonetic variants - the word as P(wi|w_i-j) (croft - look up details.)
Morphology - frozen, open, dependent, independent
syntax ...

entire 

register-language Sum(corpus)
person-language Sum(register)
idiolect-language Sum(pl) (for one language) multiple languages here and next three stages. 
dialect-language Sum(id)
language-language Sum(dia)

phylogenetically related languages (horizontal nodes) + mutation - loss
lineage related languages (from mother node, not horizontal nodes) + mutation - loss
areal languages

Disparity:
functional disparity - how different is the use? How much do speakers overlap?
morphologically related languages (not morphology/syntax, but form - mainly typology)

language death includes dialect death, not just language death. For each split, there is a simultaneous removal of homogeny in the system, and a levelling across the board towards a standard. 

x3 x3 x3 x3 x3 x3
x2 x2 x2 x2 
x1 ... .. .. .. .. <--- unlikely these dots did not also exist, and are either a) vestigially present in x3, b) extinct. 

probabilistic death / registered death. "Four score and seven years" not used *frequently any more, due to probability, not a tangible shift. 

%Possible paper: trace n-gram collocations that have died out, and explore why and how strongly.

Dunbar's number - on the one hand, it's very interesting to trace how many people we actually can have relationships with. What is important is stance and language variation among those 150 vs. outsiders. How does this work? This may be essential to defining diversity as a whole. 

Modern migration is going to make this very difficult to do accurately, and it might be best not to do that. 